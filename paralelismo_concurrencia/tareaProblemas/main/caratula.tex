

\begin{titlepage}
    %\begin{tikzpicture}[overlay, remember picture]
    %    \fill[red] (10cm,-10cm) rectangle (5cm,-15cm);
    %\end{tikzpicture}
    
    \miRectangulo{-2cm}{-4cm}{2cm}{5cm}{rosado}
%   \miRectangulo{x     }{y }{x1    }{y1   }{color}
    
    \miRectangulo{-1.5cm}{-2cm}{-1.2cm}{23.5cm}{black} % 1
    \miRectangulo{-1.8cm}{23.5cm}{5.7cm}{23.2cm}{black} % 2
    \miRectangulo{5.025cm}{23.5cm}{5.33cm}{20cm}{black} % 3
    \miRectangulo{-2.7cm}{-1.7cm}{4.7cm}{-2cm}{black} % 4
    \miRectangulo{4.2cm}{-2cm}{4.5cm}{10cm}{black} % 5

    \begin{textblock}{100}(100,20)
        \begin{flushright}
        {\huge{\textbf{Universidad Nacional del Altiplano}}}\\
        {\normalsize{\textbf{Educando mentes, Cambiando el mundo}}}
        \end{flushright}
        
    \end{textblock}
    
    \begin{tikzpicture}[remember picture, overlay]
        \node at (current page.north west) [anchor=north west, xshift=120mm, yshift=-47mm] {\includegraphics[width=0.45\textwidth]{\logoright}};
    \end{tikzpicture}
    \begin{textblock}{100}(100,130)
        \begin{flushright}
            {\Large{\textbf{Facultad de Ingeniería Mecánica Eléctrica,
                    Electrónica y Sistemas}}}\\[10pt]
            {\large{\textbf{Escuela Profesional de Ingeniería\\ de Sistemas}}}
        \end{flushright}
    \end{textblock}

    \begin{textblock}{200}(10,163)
        \begin{center}
            
            %\textcolor{azul}{\rule{\linewidth}{0.80mm}}
            % titulo del articulo
            %Monitoreo de la atención de los estudiantes mediante cámaras y celulares dentro del Aula en Puno Perú  \par
            \vspace*{\fill}
                \begin{minipage}{0.9\textwidth}
                    \centering
                    {\Large {\textbf{Problemas clasicos de Paralelismo y Concurrencia}}}\par
                \end{minipage}
            \vspace*{\fill}
            \textcolor{azul}{\rule{0.5\linewidth}{0.80mm}} \par
            \vspace{8mm}
            {\large{\textbf{ PARALELISMO, CONCURRENCIA Y SISTEMAS DISTRIBUIDOS }}} \\[10pt]
            {\large{\textbf{\textcolor{azul}{Ing. ROMERO FLORES ROBERT ANTONIO }}}} \\[20pt]
            {\large{\textbf{estudiante}}}\\[10pt]
            {\large{\textbf{$\looparrowright$   Larota Pilco David Brahyan  $\looparrowleft$ }}}\\[5pt]
            %{\large{\textbf{$\looparrowright$    Quispe Calcina Royer $\looparrowleft$ }}}\\[5pt]
            %{\large{\textbf{$\looparrowright$    Rojas Alejo Bruno $\looparrowleft$ }}}\\[5pt]
            %{\large{\textbf{$\looparrowright$  $\mathfrak{David\ Brahyan\ Larota\ Pilco}$   $\looparrowleft$ }}}\\[20pt]
            \today

        \end{center}
    \end{textblock}
\end{titlepage}
%//--------------------------------------
%@article{prueba,
%  title={prueba del documento lenguaje Latex},
%  author={Autor},
%  journal={https://www.overleaf.com/},
%  volume={13},
%  number={36},
%  pages={34--36},
%  year={2022}
%}
%
%
%\begin{lstlisting}
%
%\end{lstlisting}
%
%\begin{figure}[h]
%    \centering
%    \includegraphics[width=0.5\textwidth]{images/medicion_con_tacometro.png}
%    \caption{se realizo la medición con el tacómetro} 
%\end{figure}
%
%\begin{tabular}{ l c l }
%Tipo  			& = & 	GL-90L-4B5 \\
%Ip              & = &	55 \\
%Cos  $\varphi$    & = &  	  0.78 \\
%Voltaje         & = &	 230/400V \\
%Potencia	    & = &	2HP \\
%Intensidad    	& = & 	6.1/3.5 \\
%Frecuencia  	& = & 	60HZ \\
%Rpm     		& = &	1680 
%\end{tabular}